% Copyright 2013 Frederic Bour, all rights reserved
\documentclass{beamer}

\usepackage[french]{babel}
\usepackage[utf8x]{inputenc}
\usepackage[T1]{fontenc}
\usepackage{default}
\usepackage{tikz}

\newcommand{\sectitle}{\frametitle{\insertsection}}

\title{Merlin, un assistant pour OCaml}
\author{Frédéric \bsc{Bour}}
\date{\today}

\usetheme{Warsaw}

\AtBeginSection[] {
  \begin{frame}[plain]
    \frametitle{Plan}
    \tableofcontents[currentsection]
  \end{frame}
  \addtocounter{framenumber}{-1}
}

\newcommand{\Simley}[1]{%
\begin{tikzpicture}[scale=0.11]
    \newcommand*{\SmileyRadius}{1.0}%
    \draw [fill=brown!10] (0,0) circle (\SmileyRadius)% outside circle
        %node [yshift=-0.22*\SmileyRadius cm] {\tiny #1}% uncomment this to see the smile factor
        ;  

    \pgfmathsetmacro{\eyeX}{0.5*\SmileyRadius*cos(30)}
    \pgfmathsetmacro{\eyeY}{0.5*\SmileyRadius*sin(30)}
    \draw [fill=cyan,draw=none] (\eyeX,\eyeY) circle (0.15cm);
    \draw [fill=cyan,draw=none] (-\eyeX,\eyeY) circle (0.15cm);

    \pgfmathsetmacro{\xScale}{2*\eyeX/180}
    \pgfmathsetmacro{\yScale}{1.0*\eyeY}
    \draw[color=red, domain=-\eyeX:\eyeX]   
        plot ({\x},{
            -0.1+#1*0.15 % shift the smiley as smile decreases
            -#1*1.75*\yScale*(sin((\x+\eyeX)/\xScale))-\eyeY});
\end{tikzpicture}%
}%
\newcommand{\smiley}{\Simley{0.5}}

\begin{document}

\begin{frame}
  \titlepage
\end{frame}

% \begin{frame}{Table des matières}
%   \tableofcontents
% \end{frame}

\section{Assistant intégré à l'éditeur}

\subsection{Le toplevel standard}

\begin{frame}
  \sectitle

  Solution actuelle : le toplevel.
  \pause

  \begin{itemize}
    \item Effets de bord à l'évaluation
      \pause
    \item Phrases évaluées dans le désordre
      \pause \\
      (noms masqués, portée hasardeuse…)
  \end{itemize}
\end{frame}

\subsection{Merlin}

\begin{frame}
  \sectitle

  Ce qu'apporte Merlin.
  \pause

  \begin{itemize}
    \item Un toplevel sans interprétation, seulement du typage.
      \pause
    \item Travail incrémental, dans l'ordre du document
      (similaire à Proof-General).
      \pause
    \item Résilience aux erreurs de syntaxe et de type (expérimental).
  \end{itemize}
\end{frame}

\section{Dans la pratique…}

\subsection{Les avantages}

\begin{frame}
  \frametitle{Les avantages}

  \begin{block}{Informations de typages}
    \begin{itemize}
      \item complétion au curseur, sensible à la portée courante
        \pause
      \item type des sous-expressions
        \pause
      \item nombreuses autres analyses possibles : les fondations sont là.
        \pause
    \end{itemize}
  \end{block}

  \begin{block}{Retour immédiat}
    \begin{itemize}
      \item annotation des erreurs directement dans l'éditeur
        \pause
      \item mais attention à ne pas se laisser distraire \smiley
        \pause
    \end{itemize}
  \end{block}

  Pas de mauvaise surprise : les règles de portée, typage, etc,
  sont celles du compilateur.
\end{frame}

\subsection{Les limitations}

\begin{frame}
  \frametitle{Les limitations}
  \begin{block}{Extensions de syntaxe}
    \begin{itemize}
      \item pas de prise en charge de Camlp4 !
        \pause
      \item mais support spécifique pour certaines extensions \\
        (lwt, type-conv, …)
        \pause
    \end{itemize}
  \end{block}

  \begin{block}{Constructions difficiles à gérer}
    \begin{itemize}
      \item définitions récursives
        \pause
      \item modules de première classe, POO, etc.
        \pause
      \item[$\Rightarrow$] peu de retours sur ces constructions si le code n'est
        pas valide
    \end{itemize}
  \end{block}

\end{frame}

\subsection{Fonctionnalités}

\begin{frame}
  \sectitle

  Depuis Vim et Emacs :

  \begin{itemize}
    \item complétion,
    \item typage,
    \item annotation des erreurs,
    \item intégration avec ocamlfind \\
      {\small {\tt .merlin} pour les projets},
    \item quelques extensions de syntaxe.
  \end{itemize}

\end{frame}

\subsection{La suite}

\begin{frame}
  \frametitle{La suite}

  Dans un futur plus ou moins proche :

  \begin{itemize}
    \item travail conséquent sur les erreurs de syntaxe,
      \pause
    \item coopération avec d'autres outils (e.g. spotter, ocamldoc),
      \pause
    \item plus d'extensions. \\
      {\small js\_of\_ocaml dans une branche expérimentale}
  \end{itemize}

\end{frame}

\section*{Démo}

\begin{frame}
 \sectitle
 
 Merci pour votre attention. \\

 Pour plus d'informations : {\tt http://github.com/def-lkb/merlin} \\

 Et maintenant une petite démonstration…
 
\end{frame}

\end{document}
